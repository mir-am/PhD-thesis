\chapter*{Acknowledgments}
\addcontentsline{toc}{chapter}{Acknowledgments}
\setheader{Acknowledgments}

I am writing an unusually long acknowledgments page in an informal manner, unlike the rest of the thesis. I'll mention the names of many people who have/had been part of my PhD journey and with whom I shared so many memories and nice moments. When writing this page, I'm both happy and sad at the same time. I'm happy because I finished my PhD after 5 years, which is quite an accomplishment for me considering where I come from and the cards I had in my hand to play with. I'm sad because this chapter of my life, my PhD journey, came to an end. If I write a book about my life someday, this will be one of the best chapters to read. I've had a long PhD journey with a mixed bag of successes, failures, tears, laughs, and fears. If I could rewind time, I'd go back to live this chapter 100 times. Anyway, let's talk about my PhD journey and how it began!

In June 2019, I applied for a PhD position at the Software Engineering Research Group (SERG) of TU Delft. \textbf{Georgios Gousios} invited me for an interview. Later, he said Prof. Arie van Deursen would like to talk to you. I spoke to \textbf{Arie}. I never forget this interview in my life. He saw my Master's thesis on GitHub and asked a question about my personal blog. Suddenly, during the interview, a man driving a truck outside our apartment in Tehran was yelling with a big speaker to sell fruits. Arie asked what is this noise and I couldn't really explain it. Thankfully, they both trusted me and offered a PhD position to a young guy from thousands of kilometers away whom they had never met. In the summer of 2019, I did all the paperwork and sold my gaming PC so I could travel to the Netherlands and start my PhD journey there. After getting a temporary visa, on Oct. 10th, 2019 at around 2:20 PM, I arrived in the Netherlands and all I had was small luggage, my master's degree certificate, and a few thousand euros in my pocket.

In my first week, I met \textbf{Pouria Derakhshanfar}, a PhD student at the time, who helped me a lot with onboarding and basic things like even getting lunch since I was coming from a highly sanctioned third-world country, Iran, and I didn't have a debit/credit card to pay. I remember, one day, Pouria was wearing the Last of Us shirt, which made me play this masterpiece game later. Thank you, Pouria, for all the help in the beginning, and you are one of the few people who understand what hardships we have as an international student/immigrant. In late Oct. '19, I met \textbf{Mehdi Keshani} who became my best friend/colleague at TU Delft. I think it's kinda a life miracle that we both ended up working in the same room although Mehdi is just 3 days older than me and he's from a different city, but somehow we never crossed a path in our country but met each other in NL. Mehdi and I shared a lot of moments during our PhD at TU Delft. We laughed, joked, traveled to conferences, had fruitful collaborations, and shared painful times. Thank you, Mehdi, for all the support, understanding, and kindness, especially during the pandemic. In the first two months, I was living with some talented students, \textbf{Mehdi Asadi}, \textbf{Saieed}, \textbf{Hamid}, \textbf{Pouriya Alinaghi}. They helped me a lot in finding a place to live and settle down in Delft.

In the first few months of my PhD, everything was going nicely and I was so excited to start doing a PhD and meeting new people from all around the world. Sadly, in March 2020, a global pandemic happened and we all had to start working from home. During the pandemic, I was one of the very few people who still came to the office to work. At that time, \textbf{Joseph} was my lab/office mate. I was lucky to have him in the office so that I could chat with someone in person during the COVID lockdown. Many thanks to Joseph for all the coffee chats, lunch times, and fun discussions we had during my PhD journey. In 2020, I worked with Georgios, my daily supervisor at the time, and I was part of the Software Analytics lab (SAL). I learned many things from Georgios, including professionalism, pragmatism, efficient communication, and problem-solving. Thanks to Georgios, during the pandemic, I had the chance to hang out with the members of SAL and share good memories with them, namely, \textbf{Ayushi}, \textbf{Elvan}, Mehdi Keshani, \textbf{Maliheh (Mali)}, Joseph, and \textbf{Enrique}.

In March 2021, I started working with \textbf{Sebastian Proksch} who became my main supervisor for the rest of my PhD journey. I spent many hours with Sebastian and we had good chemistry and common interests outside work, which is kinda unusual in a typical supervisor/student relationship. We not only discussed research ideas but also talked about our hobbies like cryptocurrency mining, the latest PC games, and hardware during our coffee chats/breaks. Honestly, I was fortunate to have a supervisor like him who is a PC gamer like myself. Special Thanks to Sebastian for all his direct in-depth feedback and helpful discussions. I hope he'll forget my complaints, laments, and arguments, especially at the end of my PhD journey. Though, I still have one complaint, which is he didn't fulfill my wish, an Nvidia RTX 5090 as a graduation gift. Joking aside, without his massive support and guidance, I would never finish this PhD. 

In 2021, I also met PhD students, \textbf{Ali Khatami}, \textbf{Aru}, \textbf{Mark}, and \textbf{Imara} who joined SERG. I chatted with them many times during our breaks and we shared many good memories at conferences and summer schools. I remember Ali Khatami once approached me to help him with an issue on Linux and I almost erased the whole root filesystem! This is how I also became a Linux admin during my PhD! Ali Khatami is also a cool person to hang out with but, unfortunately, we never went to a concert/festival together, maybe, one day, we'll go somewhere. I know Aru was probably tired of me doing hand hugs with him every time we saw each other at the office. Also, I'm looking forward to having a coffee/drink again with Aru someday in Amsterdam. I know Mark for being very social and keen to learn about other different cultures. I remember walking fast with Imara to the TUD library to grab a "premium" coffee and it was certainly good sport during the working day!

In 2022, I met \textbf{Amir Deljouyi}, \textbf{Shujun}, \textbf{Ali Al-kaswan}, \textbf{Baris}, \textbf{June}, and \textbf{Lorena}. All of them are nice supportive colleagues I had at SERG. Amir is humble and we made jokes and laughed many times at the office. Also, Amir is good at EA FIFA and he annihilated my team several times in this game. One day, I will take revenge! I remember Shujun as one of my office/lab mates and she is very polite and was a Dota 2 player like myself. Ali Al-kaswan was also one of my office mates and friends who tolerated my corny jokes and random chats. Ali and I talked a lot about PC gaming and hardware, and we traveled to very far places in the world like Australia and we shared many nice memories. I definitely miss Ali Al-kaswan as my office mate. I'd like to mention Baris with whom I had good times in Delft. He is a funny cool person, a gamer like myself, had fun at his place several times, and made me laugh many times. If I wanted to invite people to my place, he would be number one on the list. I owe him a nice dinner. June was a very supportive and nice colleague of mine at SERG. She was exactly the postdoc a PhD student needs in hard times to motivate them. She was also great at cracking harsh jokes. Also, I know Lorena for being super polite and her warm personality. I still miss the special Romanian drink she used to share with us.

In 2023, I became colleagues with \textbf{Jonathan}, \textbf{Berkay}, \textbf{Anthony}, \textbf{Aral}, and \textbf{Sara Regali}. Jonathan was also my former Master's student and we became an office mate. He is a humble, dedicated hard-working person and an ML enthusiast. I know Anthony for talking about souls-like games like Elden Ring and how to beat these games. He's also a cool hard-working person, whom I'd like to hang out with. Berkay was one of the few PC gamers at SERG with whom I had coffee chats about games. Yet, I don't know why Berkay is a bit hesitant to say hi to me when we see each other outside the university. Aral was one of the brilliant bachelor students I met at TUD. Aral and I were office mates and we used to play Nintendo games together during our break time. I never had a chance to win a game against Aral. Sara was also visiting SERG to finish her Master's thesis. She was funny and cool. I hope she still remembers what I told her, "Don't forget your former colleague after getting a job in Amsterdam!". I want to mention \textbf{Roham} who is a smart and dedicated bachelor student like Aral. If you, Roham, are reading this, keep up the good work and be consistent. You can achieve 10x what I've done in my life so far when you reach my age, given your solid foundation and the circle of people around you. 

I'd like to mention \textbf{Mitchell}, \textbf{Carol}, \textbf{Leonhard}, the former PhD students at SERG, whom I've known since 2020. I had good memories with all of them, from the summer school in Cordoba, the IPA events, and the SE conferences we attended. During our PhD journeys, we talked about various things and laughed in different places, at coffee machines, bars, restaurants, and zoo. I remember convincing Leonhard that Cyberpunk 2077 was not that buggy at launch! Unfortunately, I haven't talked to him over the past two years but I wish him the best with his new adventure in Singapore. 

After mentioning my peers, I'd like to thank the professors during my time at SERG. First, I thank \textbf{Andy} who, at times, used to challenge me with his critical questions during my presentations. Also, Andy supported my content and posts on social media many times, although he had no obligation to do so. \textbf{Annibale} might be the coolest professor I've ever met in my academic career. I can't count how many times he made us laugh and he was one of the few colleagues I could talk to at the office during the COVID pandemic. Also, his brother, \textbf{Sebastiano} is a great researcher and he was one of the cool people I used to hang out with at the SE conferences/events.  I remember \textbf{Luis} also arrived in NL one month before me and we both knew how it feels to start a new job in another country where you barely know anyone. Although Luis' jokes made me feel a little bit embarrassed at times, he was definitely supportive in hard times. Also, thanks to \textbf{Diomodis} for his great online course on Unix and for inviting me to become a social media co-chair at the MSR'24 conference. I remember \textbf{Thomas} for the coffee chats, grabbing lunch outside, and we also built a PC together at the office. \textbf{Maurício} was always kind and he is also one of the stars in the industry I look up to. Together, we also supervised Jonathan's Master's thesis. I remember \textbf{Burcu} for being super polite, coffee chats, and delicious Turkish delights. Also, many thanks to \textbf{Mali} for inviting me to her research lab's social events. I’ve met very few ambitious hard-working people like Mali. Unfortunately, I didn't have the chance to work with her given that we have similar research interests. Maybe, we'll collaborate in the future.

Also, I'd like to mention the people I know at JetBrains, \textbf{Vladimir}, \textbf{Yaroslav}, and Pouria. They are all very nice and cool people and I hung out with them at the SE conferences. They're so kind to answer my messages quickly on social media despite being super busy. If I had a chance to work at JetBrains, I'd try to join their teams. During my PhD, I worked on the FASTEN project and I had the pleasure of collaborating with SIG, namely, \textbf{Magiel}, \textbf{Chushu}, and \textbf{Miroslav}. They are all professional people, competent industry practitioners, and good at doing business. I also thank \textbf{Paige Bailey} (then-Product Manager of GitHub) for promoting the VSCode extension of Type4Py on X (Twitter), discussing it with her team at Microsoft, and giving us feedback. This helped us to attract Python developers to try the extension before submitting Type4Py's paper to ICSE'22.

I'd like to thank \textbf{Ashwin} whom I first met at the SANER'23 conf. in Macau. We collaborated on ML and software analysis and published two papers. Aside from research, he's also a cool person to hang out with. I'm still looking forward to smoking a cigar with him to celebrate our collaboration. 

Also, I had the pleasure of supervising two wonderful students, \textbf{Evaldas} (bachelor) and \textbf{Lang} (master). I didn't just supervise them but I learned things from them. I'm happy that they're both working on fancy stuff at big tech companies.

A big shout out to the management assistance people at SERG, \textbf{Minaksi} and \textbf{Kim}. They were both kind and handled my inquiries smoothly and efficiently like visa extensions, business travels, paperwork, etc. I still remember Minaksie knocking hard on our door and asking us to join her for a coffee break. Many thanks to both of them.

During my PhD, I was still in touch with my high school friends, \textbf{Morteza} and \textbf{Farzan}. We laughed, yelled, cried, and fought while playing Dota 2. They won't forget the infamous moment I died as Wraith King three times in a row during a match of Dota 2 and I said whatever bad words I knew in Persian! I remember coming back home from work but these two guys never let me feel tired with their jokes and memes. If I go back to Iran one day, I'll visit them to have a high school reunion. Also, shout out to \textbf{Pouya}, a friend of mine, with whom I played Counter Strike 2 and God knows how many players we bothered with our nonsense sayings in the game. People won't forget our ridiculous fights over looting stuff in Arena: Breakout Infinite on your live stream. I remember my "old" friends, \textbf{Ramin} and \textbf{Vahid}, for their emotional support and our daily chats about tech and PC overclocking/benchmarking.

I was also part of the IPA PhD council during my PhD and I would like to thank my peers there, namely, \textbf{Niels}, \textbf{Tom}, \textbf{Ivo}, \textbf{Philip}, \textbf{Christopher}, and \textbf{Lieuwe}. We organized fun social events for the IPA Fall Days. I thank \textbf{Loek}, the then-managing director of the IPA research school for his support and participation in our social games.

I shall mention my aunt, \textbf{Azadeh}, whom I visited many times in Germany during my PhD. It was great to have my aunt close to me and she was like my mother outside Iran. She cooked my favorite Iranian food and was so kind to give me gifts and money, which I didn't need. Also, shout to my cousins \textbf{Sarah} and \textbf{Hossien} with whom I grew up and it was always nice to see them in Germany. It's very sad that my relationship with my aunt was recently sabotaged by some impulsive people in our family.

I can't finish this acknowledgment without thanking \textbf{Arie}, my big boss at TUD. I never forget his massive support during my PhD such as contract extensions and funding my travels. He's a great example of leadership for me. It would've been awesome if I could work with him as a postdoc researcher. 

I'd like to mention the names of my former colleagues at SERG with whom I shared nice moments, from a short coffee chat, to attending local events in NL, and traveling to conferences. I can write sentences about every one of them, but this acknowledgment will become a book in itself. Thanks to \textbf{Fabio}, \textbf{Dimitri}, \textbf{Jean}, \textbf{Vivek}, \textbf{Marielli}, \textbf{Wouter}, \textbf{Xavier}, \textbf{Davide}, \textbf{Xunhui}, \textbf{Gemma}, \textbf{Luca}, \textbf{Anand}, \textbf{Moritz}, \textbf{Quentin}, \textbf{João}, \textbf{Eileen}.

At the end, I'd like to mention the names of the cool PhD students whom I met and hung out with at SE conferences and summer schools. They were doing interesting research. Thanks to \textbf{Sajad}, \textbf{Kevin}, \textbf{Nafiseh}, \textbf{Nathan}, \textbf{Satrio}, \textbf{Cezar}, \textbf{Mahtab (Mattie)}, \textbf{Christian}, \textbf{Pooya Rostami}, \textbf{Gunnar}.

To wrap this up, I thank the independent members of my PhD committee, \textbf{Fernando Kuipers}, \textbf{
Michael Pradel}, \textbf{Prem Devanbu}, and \textbf{Baishakhi Ray} for examining my PhD thesis and their invaluable feedback.

Wait a second! I thanked many people here but myself! I worked hard to reach this point in my life where I have the luxury of obtaining a PhD degree. But this is not the end of the movie (I hope!). The next chapter has already begun for me. However, the picture I have for my future in my mind is not crystal clear. In life, you never truly know where your boat might crash or arrive in one piece in a stormy ocean. Life is generally way more complex than the research problems I addressed in this thesis. Although one may do every step perfectly, life can still surprise them in a way that they didn't expect. I'm not a perfect human and had "mistakes" that I could probably avoid (you're never gonna find a mistake in this thesis!). I should probably "punish" myself for the things I did wrong, instead of blaming others or expecting people to do me a favor! As the cliché goes, your biggest "enemy" is the person in the mirror. There are many aspects in doing a PhD and one of them is doing novel research. Some say doing a PhD is a lonely journey but it's not! I can't deny that the people around you will have an influence on you (in)directly and little or big. I don't wanna lecture you here about what you should do right in a PhD journey. It's just beyond the scope of this thesis. In the near future, I should write a blog post about my mistakes and failures during this journey. On the other hand, If the concept of destiny exists, there were probably no mistakes in things I did in my life. Maybe, I was supposed to do whatever I did and there are no alternative life paths for me. If I could go back in time, I'd probably end up writing this sentence again. This is also a reason why I'm not really jealous of anyone. They may live their own life path and I do mine. I'm genuinely happy that I met all of these people mentioned in this section. There are billions of other people on the Earth whom I'm never gonna meet. Somehow, my life path crossed theirs. All of them have their own story and certainly, they have something to teach me. Regardless of destiny, I hope you won't see me soon saying "How the heck I ended up here! This ain't supposed to be my future!". Anyhow, these are some of my thoughts on life, and thanks for reading my ramblings/rants. Of course, you can ignore whatever I said. Perhaps, the most important lesson I learned by doing a PhD was to think critically for myself before accepting blindly whatever I read or hear.

\begin{flushright}
{\makeatletter\itshape
    Amir \\
    Delft,  January 2025
\makeatother}
\end{flushright}


