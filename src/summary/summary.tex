\chapter*{Summary}
\addcontentsline{toc}{chapter}{Summary}
\setheader{Summary}

Software engineering, fundamental to modern technological advancement, profoundly influences various aspects of society by enhancing efficiency, accessibility, and security. This discipline involves systematically applying engineering principles to software systems' design, development, testing, and maintenance. Innovations in software engineering have revolutionized industries such as communication, finance, healthcare, and education, democratizing access to information and connecting global communities. As software systems become increasingly complex, the need for efficient, secure, and reliable software analysis tools becomes paramount.

The thesis focuses on improving the actionability and scalability of software analysis by integrating machine learning (ML) techniques. Traditional static analysis tools often struggle with large codebases, leading to high false positive rates and high computational costs. Machine learning, particularly deep learning architectures like Transformers, offers a promising solution by capturing long-range dependencies in code and learning hierarchical representations. This capability enables ML models to automate tasks such as bug detection, source code summarization, and program repair, providing developers with actionable insights and improving overall productivity and code quality.

A significant contribution of this thesis is the development of ML-based techniques for type inference in Python and call graph pruning. An ML-based type inference approach, namely Type4Py, was proposed, which accurately predicts type annotations for Python code, enhancing code quality and reducing runtime errors. ML models with conservative pruning strategies were proposed for call graph pruning, which learns from dynamic traces obtained by executing programs to identify and eliminate false edges, thereby minimizing false positives and improving precision. Additionally, the thesis explores the application of call graphs in vulnerability analysis, demonstrating that granular assessments provide more accurate and actionable insights than more straightforward, dependency-level analyses.

In summary, this thesis advances the field of software analysis by harnessing machine learning to address two important issues related to the actionability and scalability of software analysis tools. The proposed ML-driven tools and techniques enhance the precision and reliability of software analysis and support developers in maintaining robust, secure, and maintainable software systems. These contributions pave the way for future research in applying ML techniques to various aspects of software engineering, promising further improvements in software development practices.


\chapter*{Samenvatting}
\addcontentsline{toc}{chapter}{Samenvatting}
\setheader{Samenvatting}

{\selectlanguage{dutch}
Software engineering, fundamenteel voor moderne technologische vooruitgang, beïnvloedt diepgaand verschillende aspecten van de samenleving door efficiëntie, toegankelijkheid en veiligheid te verbeteren. Deze discipline omvat het systematisch toepassen van technische principes op het ontwerp, de ontwikkeling, het testen en het onderhoud van softwaresystemen. Innovaties in software engineering hebben industrieën zoals communicatie, financiën, gezondheidszorg en onderwijs gerevolutioneerd, toegang tot informatie gedemocratiseerd en wereldwijde gemeenschappen verbonden. Naarmate softwaresystemen steeds complexer worden, wordt de behoefte aan efficiënte, veilige en betrouwbare software-analysetools steeds belangrijker.

De scriptie richt zich op het verbeteren van de bruikbaarheid en schaalbaarheid van software-analyse door integratie van machine learning (ML) technieken. Traditionele statische analysetools hebben vaak moeite met grote codebases, wat leidt tot hoge foutpositieve percentages en hoge computatiekosten. Machine learning, met name deep learning-architecturen zoals Transformers, biedt een veelbelovende oplossing door langeafstandsafhankelijkheden in code vast te leggen en hiërarchische representaties te leren. Deze mogelijkheid stelt ML-modellen in staat om taken zoals bugdetectie, broncode-samenvatting en programmareparatie te automatiseren, waardoor ontwikkelaars bruikbare inzichten krijgen en de productiviteit en codekwaliteit in het algemeen verbeteren.

Een belangrijke bijdrage van deze scriptie is de ontwikkeling van ML-gebaseerde technieken voor type-inferentie in Python en call graph pruning. Een ML-gebaseerde type-inferentiebenadering, namelijk Type4Py, werd voorgesteld, die nauwkeurig typeannotaties voor Python-code voorspelt, de codekwaliteit verbetert en runtime-fouten vermindert. ML-modellen met conservatieve snoeistrategieën werden voorgesteld voor call graph pruning, die leren van dynamische traceringen verkregen door programma's uit te voeren om valse randen te identificeren en te elimineren, waardoor foutpositieven worden geminimaliseerd en de precisie wordt verbeterd. Daarnaast onderzoekt de scriptie de toepassing van call graphs in kwetsbaarheidsanalyse, waarbij wordt aangetoond dat gedetailleerde beoordelingen nauwkeurigere en bruikbaardere inzichten bieden dan eenvoudigere, afhankelijkheidsniveau-analyses.

Samenvattend, deze scriptie bevordert het veld van software-analyse door machine learning te gebruiken om twee belangrijke problemen met betrekking tot de bruikbaarheid en schaalbaarheid van software-analysetools aan te pakken. De voorgestelde ML-gedreven tools en technieken verbeteren de precisie en betrouwbaarheid van software-analyse en ondersteunen ontwikkelaars bij het onderhouden van robuuste, veilige en onderhoudbare softwaresystemen. Deze bijdragen effenen het pad voor toekomstig onderzoek naar de toepassing van ML-technieken op verschillende aspecten van software engineering, wat verdere verbeteringen in softwareontwikkelingspraktijken belooft.
}



